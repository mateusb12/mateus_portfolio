\documentclass[letterpaper,11pt]{article}

\usepackage{latexsym}
\usepackage[empty]{fullpage}
\usepackage{titlesec}
\usepackage{marvosym}
\usepackage[usenames,dvipsnames]{color}
\usepackage{verbatim}
\usepackage{enumitem}
\usepackage[hidelinks]{hyperref}
\usepackage{fancyhdr}
\usepackage[brazilian]{babel} % Alterado para hifenização correta em PT-BR
\usepackage{tabularx}
\usepackage{fontawesome5}
\usepackage{multicol}
\setlength{\multicolsep}{-3.0pt}
\setlength{\columnsep}{-1pt}
\input{glyphtounicode}
\usepackage[margin=1.4cm]{geometry}

\pagestyle{fancy}
\fancyhf{}
\fancyfoot{}
\renewcommand{\headrulewidth}{0pt}
\renewcommand{\footrulewidth}{0pt}

\addtolength{\oddsidemargin}{-0.15in}
\addtolength{\textwidth}{0.3in}

\urlstyle{same}

\raggedbottom
\raggedright
\setlength{\tabcolsep}{0in}

% Sections formatting
\titleformat{\section}{
    \vspace{-4pt}\scshape\raggedright\large\bfseries
}{}{0em}{}[\color{black}\titlerule \vspace{-5pt}]

\pdfgentounicode=1

%-------------------------
% Custom commands
\newcommand{\resumeItem}[1]{
    \item\small{
            {#1 \vspace{-2pt}}
    }
}

\newcommand{\resumeSubheading}[4]{
    \vspace{-2pt}\item
    \begin{tabular*}{1.0\textwidth}[t]{l@{\extracolsep{\fill}}r}
    \textbf{#1} & \textbf{\small #2} \\
    \textit{\small#3} & \textit{\small #4} \\
    \end{tabular*}\vspace{-7pt}
}

\newcommand{\resumeProjectHeading}[2]{
    \item
    \begin{tabular*}{1.0\textwidth}{l@{\extracolsep{\fill}}r}
    \small#1 & \textbf{\small #2}\\
    \end{tabular*}\vspace{-7pt}
}

\newcommand{\resumeSubHeadingListStart}{\begin{itemize}[leftmargin=0.0in, label={}]}
\newcommand{\resumeSubHeadingListEnd}{\end{itemize}\vspace{0pt}}
\newcommand{\resumeItemListStart}{\begin{itemize}}
\newcommand{\resumeItemListEnd}{\end{itemize}\vspace{-5pt}}

\begin{document}

    %----------HEADING----------
    \begin{center}
    {\Large \scshape Mateus Bessa} \\[2mm]
    \footnotesize
    \faPhone\ \underline{+55 (85) 99917-1902} ~
    \faEnvelope\ \underline{matbessa12@gmail.com} ~
    \faLinkedin\ \underline{\href{https://www.linkedin.com/in/mateus-bessa-m/}{linkedin.com/in/mateus-bessa-m}}
    \\[1mm]
    \faGithub\ \underline{\href{https://github.com/mateusb12}{github.com/mateusb12}} ~
    \faBriefcase\ \underline{\href{https://mateusb12.github.io/mateus_portfolio/}{mateusb12.github.io}}
    \end{center}

    %-----------Experience---------------
    \section{Experiência Profissional}
    \resumeSubHeadingListStart

    \resumeSubheading{RovesterAI}{Ago 2025 -- Atualmente}{Desenvolvedor Full-stack}{Brasil}
    \resumeItemListStart
    \resumeItem{Projetei e implantei pipelines de telemetria em \textbf{mais de 20 drones Raspberry Pi}, permitindo coordenação de alimentação e diagnósticos em tempo real sob condições de baixa largura de banda.}
    \resumeItem{Integrei sensores de corrente \textbf{ADS1x15} a um dashboard leve (HTML/CSS/JS), reduzindo diagnósticos manuais em campo em \textbf{~70\%}.}
    \resumeItem{Reduzi falhas de entrega de mensagens dos drones em \textbf{~40\%} através de políticas de retry e acknowledgements via \textbf{RabbitMQ}.}
    \resumeItem{Implementei fluxos de recuperação autônoma usando \textbf{APScheduler}, cron e watchdogs, eliminando a necessidade de resets manuais diários via SSH.}
    \resumeItem{Desenvolvi \textbf{mais de 8 APIs backend} para listagem de fazendas, presença de dispositivos, estados de alimentação e notificações push.}
    \resumeItem{Reduzi o uso de RAM nos agentes dos drones em \textbf{35\%} através de otimização em tempo de execução em ambientes Alpine/Linux.}
    \resumeItem{Stack: \textbf{FastAPI}, \textbf{RabbitMQ}, \textbf{MongoDB}, \textbf{SQLite}, \textbf{Docker}, \textbf{AWS EC2}, \textbf{React Native}.}
    \resumeItemListEnd

    \resumeSubheading{Pontotel}{Jun 2024 -- Jun 2025}{Backend Developer}{Brazil}
    \resumeItemListStart
    \resumeItem{Construí scripts de automação baseados em GCP, reduzindo o tempo de resposta para solicitações de clientes de \textbf{horas para minutos}.}
    \resumeItem{Refatorei lógica de validação legada utilizando \textbf{DTOs}, reduzindo bugs relacionados a validação em \textbf{~40\%}.}
    \resumeItem{Desenvolvi importadores com etapas de pré-validação para prevenir a ingestão de dados inconsistentes.}
    \resumeItem{Projetei API interna para monitoramento de calendários usando métricas do \textbf{MongoDB}, reduzindo o tempo de depuração em \textbf{~50\%}.}
    \resumeItem{Stack: \textbf{Flask, Celery, FastAPI, TILT, Alembic, Poetry, Linux, Docker, Kubernetes}.}
    \resumeItemListEnd

    \resumeSubheading{Omnichat Startup}{Dec 2023 -- May 2024}{Backend Lead}{Brazil}
    \resumeItemListStart
    \resumeItem{Coordenei sprints da equipe de backend e forneci mentoria/code reviews.}
    \resumeItem{Desenvolvi APIs RESTful em Flask com arquitetura de microsserviços e autenticação JWT.}
    \resumeItem{Gerenciei modelagem e otimização de banco de dados PostgreSQL.}
    \resumeItem{Adotei princípios de Clean Code e documentei APIs via Swagger.}
    \resumeItemListEnd

    \resumeSubheading{Insane Games}{Jul 2021 -- Feb 2022}{Intern}{Brazil}
    \resumeItemListStart
    \resumeItem{Desenvolvi sistemas de backend para jogos Unity em C\#.}
    \resumeItem{Gerenciei o processo de produção de assets em Tech Art utilizando Blender.}
    \resumeItemListEnd

    \resumeSubHeadingListEnd


    %-----------PROJECTS-----------
    \section{Projetos}
    \resumeSubHeadingListStart

    % --- Investments Calculator ---
    \resumeProjectHeading
    {\textbf{Investments Calculator} $|$ \emph{React, Vite, Tailwind, Supabase}}
    {\emph{\href{https://mateusb12.github.io/investments-calculator}{Website}}}
    \resumeItemListStart
    \resumeItem{Aplicação web React que fornece múltiplas calculadoras financeiras para o mercado brasileiro (FIIs, CDBs, LCIs).}
    \resumeItemListEnd

    % --- Book Analyzer ---
    \resumeProjectHeading
    {\textbf{Book Analyzer} $|$ \emph{NLP, Django, Python, Pandas, SpaCy}}
    {\emph{\href{https://github.com/mateusb12/WitcherAnalysis}{GitHub}}}
    \resumeItemListStart
    \resumeItem{Transforma livros brutos (\texttt{.txt}) em grafos estruturados estilo rede social usando pipelines de NLP.}
    \resumeItemListEnd

    % --- Flight Scraper ---
    \resumeProjectHeading
    {\textbf{Flight Scraper} $|$ \emph{Flask, Firebase, Selenium, Telegram}}
    {\emph{\href{https://github.com/mateusb12/TravelScrapper}{GitHub}}}
    \resumeItemListStart
    \resumeItem{Realiza raspagem de dados de voos e envia alertas automatizados via Telegram quando os preços caem.}
    \resumeItemListEnd

    % --- Valorant Impact ---
    \resumeProjectHeading
    {\textbf{Valorant Impact} $|$ \emph{ML, Flask, LightGBM, Scikit-Learn}}
    {\emph{\href{https://github.com/mateusb12/valorant_impact}{GitHub}}}
    \resumeItemListStart
    \resumeItem{Quantifica ações de jogadores sobre a probabilidade de vitória usando modelos de ML otimizados com Optuna.}
    \resumeItemListEnd

    \resumeSubHeadingListEnd


    %-----------EDUCATION-----------
    \section{Formação Acadêmica}
    \resumeSubHeadingListStart
    \resumeSubheading
    {Universidade de Fortaleza}{2019 -- 2023}
    {Bacharelado em Ciência da Computação}{Fortaleza, Brasil}

    \resumeSubheading
    {Politechnika Lubelska}{2022}
    {Erasmus / Intercâmbio Acadêmico}{Lublin, Polônia}
    \resumeSubHeadingListEnd


    %-----------TECHNICAL SKILLS-----------
    \section{Habilidades Técnicas}
    \begin{itemize}[leftmargin=0.15in, label={}]
        \small{\item{
            \textbf{Linguagens}{: Python, C\#, JavaScript, TypeScript, SQL, Bash} \\
            \textbf{Frameworks}{: Flask, Django, FastAPI, Node.js, React, React Native, NextJS} \\
            \textbf{Cloud \& Infra}{: AWS, Google Cloud, Azure, Docker, Kubernetes, TILT, GitHub Actions, Linux} \\
            \textbf{Banco de Dados}{: PostgreSQL, MongoDB, Supabase} \\
            \textbf{Conceitos \& Ferramentas}{: Scrum, Agile, Kanban, JWT, OAuth2, Swagger, Unit Testing} \\
            \textbf{Idiomas}{: Português (Nativo), Inglês (C2 Fluente), Espanhol (Básico)}
        }}
    \end{itemize}

    %-----------LANGUAGES-----------
    \section{Idiomas}

    \begin{itemize}[leftmargin=0.15in, label={}]
        \small{\item{
            \textbf{Inglês}: C2 Fluente \\
            \textbf{Português}: C2 Nativo \\
            \textbf{Espanhol}: A1 Básico
        }}
    \end{itemize}

\end{document}